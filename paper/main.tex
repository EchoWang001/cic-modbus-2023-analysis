\documentclass[conference]{IEEEtran}
\IEEEoverridecommandlockouts

\usepackage{cite}
\usepackage{amsmath,amssymb,amsfonts}
\usepackage{algorithmic}
\usepackage{graphicx}
\usepackage{textcomp}
\usepackage{xcolor}
\usepackage{booktabs}
\usepackage{multirow}
\usepackage{array}

\def\BibTeX{{\rm B\kern-.05em{\sc i\kern-.025em b}\kern-.08em
    T\kern-.1667em\lower.7ex\hbox{E}\kern-.125emX}}

\begin{document}

\title{Feature-based Anomaly Detection in Modbus Networks: A Comprehensive Study on CIC Dataset 2023}

\author{\IEEEauthorblockN{WANG YUJIE and WANG ZHAOHUI*}
\IEEEauthorblockA{\textit{CHN ENERGY ZHISHEN CONTROL TECHNOLOGY CO.,LTD} \\
Beijing, China \\
Email: \{wyujie000@163.com, 12071908@chnenergy.com.cn\}}
\thanks{*Corresponding author.}
}

\maketitle

%==============================================================================
% ABSTRACT
%==============================================================================
\begin{abstract}
The security and resilience of industrial control systems (ICS) have become critical concerns as Modbus-based networks remain vulnerable to cyber attacks due to the protocol's lack of built-in authentication and encryption. This paper presents a comprehensive feature-based analysis of the CIC Modbus Dataset 2023 for anomaly detection in industrial networks. We design 44 features across protocol, temporal, and operation pattern categories, and systematically evaluate five machine learning algorithms against a rule-based baseline. Our analysis reveals that the feature consecutive\_write\_max effectively captures attack patterns, achieving perfect separation between attack and normal traffic in SCADA and IED scenarios. Experimental results demonstrate that XGBoost achieves the best F1-score of 96.17\% with inference time of only 0.001 ms per sample, making it suitable for real-time industrial deployment. Compared to the simple rule baseline with 100\% recall but 67.64\% precision, machine learning methods reduce false positives by 65-80\%, improving operational efficiency in industrial monitoring. We provide practical deployment guidelines for different industrial scenarios and discuss implications for ICS security research.
\end{abstract}

\begin{IEEEkeywords}
Industrial control systems, Modbus protocol, anomaly detection, feature engineering, machine learning, intrusion detection systems (IDS), SCADA security
\end{IEEEkeywords}

%==============================================================================
% I. INTRODUCTION
%==============================================================================
\section{Introduction}

The reliability, security, and resilience of industrial control systems (ICS) have become paramount concerns in the era of Industry 4.0 \cite{ref_iiot_security}, \cite{ref_iiot_edge}. Modern industrial infrastructures, including power grids, manufacturing plants, and water treatment facilities, rely heavily on Supervisory Control and Data Acquisition (SCADA) systems and Distributed Control Systems (DCS) for monitoring and control operations.

Among industrial communication protocols, Modbus remains one of the most widely deployed due to its simplicity and interoperability \cite{ref_modbus_model}. However, designed in 1979, Modbus lacks fundamental security mechanisms such as authentication, encryption, and integrity verification \cite{ref_modbus_security}. This inherent vulnerability exposes Modbus-based industrial systems to various cyber attacks including reconnaissance, denial of service, unauthorized write operations, and replay attacks \cite{ref_modbus_rules}.

The Canadian Institute for Cybersecurity released the CIC Modbus Dataset 2023 \cite{ref_cic_dataset}, providing researchers with a valuable resource for developing and evaluating anomaly detection methods for industrial networks. However, systematic analysis of this dataset's characteristics and the effectiveness of different detection approaches remains limited.

This paper addresses the following research questions:
\begin{itemize}
    \item What are the attack patterns and feature distributions in the CIC Modbus 2023 dataset?
    \item Which features are most effective for detecting anomalies in Modbus networks?
    \item How do machine learning methods compare to rule-based approaches in terms of detection performance and computational efficiency?
    \item What are the practical implications for deploying anomaly detection in industrial environments?
\end{itemize}

The main contributions of this paper are:
\begin{enumerate}
    \item \textbf{Comprehensive Dataset Analysis}: We provide the first systematic feature-based analysis of the CIC Modbus 2023 dataset, revealing its attack pattern characteristics and data distributions.
    \item \textbf{Feature Engineering}: We design 44 features across protocol, temporal, and operation pattern categories, identifying \texttt{consecutive\_write\_max} as the critical discriminative feature.
    \item \textbf{Method Comparison}: We systematically evaluate rule-based, traditional machine learning, and neural network approaches, quantifying their trade-offs in detection performance and computational cost.
    \item \textbf{Practical Deployment Guidelines}: We provide recommendations for selecting detection methods based on industrial operational requirements, balancing security and efficiency.
\end{enumerate}

The remainder of this paper is organized as follows. Section II reviews related work. Section III describes the dataset and feature engineering. Section IV presents our methodology. Section V reports experimental results. Section VI discusses findings and limitations. Section VII concludes the paper.

%==============================================================================
% II. RELATED WORK
%==============================================================================
\section{Related Work}

\subsection{ICS/SCADA Security Threats}
Industrial control systems have become increasingly targeted by sophisticated cyber attacks \cite{ref_iiot_security}. Notable incidents include Stuxnet \cite{ref_stuxnet}, which targeted Iranian nuclear facilities, and BlackEnergy \cite{ref_blackenergy}, which caused power outages in Ukraine. These events highlight the critical need for effective anomaly detection in industrial environments. The convergence of operational technology (OT) and information technology (IT) has expanded the attack surface of ICS networks \cite{ref_iiot_edge}.

\subsection{Modbus Protocol Security}
% Discuss Modbus vulnerabilities and attack types
The Modbus protocol operates without authentication or encryption, making it vulnerable to various attack vectors. Common attack types include:
\begin{itemize}
    \item \textbf{Reconnaissance}: Scanning for active devices and function codes
    \item \textbf{Denial of Service}: Overwhelming devices with malformed requests
    \item \textbf{Write Attacks}: Unauthorized modification of control parameters
    \item \textbf{Replay Attacks}: Retransmitting captured legitimate commands
\end{itemize}

Several researchers have proposed Modbus-specific intrusion detection systems. Goldenberg and Wool \cite{ref_modbus_security} developed deterministic finite automaton (DFA) based models for accurate Modbus/TCP traffic characterization, achieving high detection rates with low false positives. Kleinmann and Wool \cite{ref_wool} extended this approach to model the Siemens S7 SCADA protocol. Hong et al. \cite{ref_modbus_rules} proposed integrated anomaly detection methods combining signature-based and anomaly-based approaches for substation cyber security. Gao and Morris \cite{ref_modbus_model} developed signature-based intrusion detection for Modbus-based industrial control systems.

\subsection{ML-based Anomaly Detection for ICS}
Machine learning approaches have been extensively applied to ICS anomaly detection \cite{ref_ics_ml}. Garcia-Teodoro et al. \cite{ref_ics_ml} provided a comprehensive survey of anomaly-based network intrusion detection techniques. Upadhyay and Sharma \cite{ref_ml_ids} reviewed network-based SCADA intrusion detection systems, highlighting the effectiveness of supervised learning methods.

Traditional methods such as Random Forest and XGBoost have shown strong performance in ICS intrusion detection. Zhu et al. \cite{ref_rf_ics} optimized Random Forest hyperparameters for power ICS intrusion detection, achieving 98\% accuracy. Le et al. \cite{ref_xgboost} applied XGBoost to imbalanced IIoT datasets, demonstrating F1-scores exceeding 99\%.

Deep learning methods have been explored for capturing temporal dependencies in SCADA traffic. Potluri and Diedrich \cite{ref_dl_ids} proposed accelerated deep neural networks for enhanced IDS. Radoglou-Grammatikis and Sarigiannidis \cite{ref_lstm_scada} compiled intrusion detection approaches for smart grid security. Hadžiosmanović et al. \cite{ref_feature_eng} introduced semantic security monitoring using process-level features.

Despite these advances, systematic analysis of the recently released CIC Modbus 2023 dataset \cite{ref_cic_dataset} remains limited, motivating our comprehensive study that combines feature analysis with lightweight detection methods suitable for resource-constrained IIoT environments.

%==============================================================================
% III. DATASET AND FEATURE ENGINEERING
%==============================================================================
\section{Dataset and Feature Engineering}

\subsection{CIC Modbus Dataset 2023}

\subsubsection{Dataset Overview}
The CIC Modbus Dataset 2023 \cite{ref_cic_dataset} was generated using a Docker-based SCADA testbed simulating a power grid environment. The dataset contains network traffic captured in PCAP format across four scenarios: Benign, Compromised-SCADA, Compromised-IED, and External attack.

Table \ref{tab:dataset_stats} summarizes the dataset statistics after preprocessing with 15-second time windows and minimum 30-packet threshold filtering.

\begin{table}[htbp]
\caption{Dataset Statistics}
\begin{center}
\begin{tabular}{lrrrrr}
\toprule
\textbf{Scenario} & \textbf{Files} & \textbf{Windows} & \textbf{Normal} & \textbf{Attack} \\
\midrule
Benign & 19 & 44,821 & 44,821 & 0 \\
Compromised-SCADA & 17 & 38,799 & 33,681 & 5,118 \\
Compromised-IED & 6 & 34,582 & 34,483 & 99 \\
External & 1 & 7 & 6 & 1 \\
\midrule
\textbf{Total} & \textbf{43} & \textbf{118,209} & \textbf{112,991} & \textbf{5,218} \\
\bottomrule
\end{tabular}
\label{tab:dataset_stats}
\end{center}
\end{table}

\subsubsection{Labeling Strategy}
Due to limitations in the provided CSV labels, we adopt a hybrid labeling strategy:
\begin{itemize}
    \item \textbf{External scenario}: Windows containing packets from attacker IP (185.175.0.7) are labeled as Attack
    \item \textbf{Compromised scenarios}: Windows containing write operations (Function Codes 5, 6, 15, 16) are labeled as Attack
\end{itemize}

This strategy aligns with the dataset's attack characteristics, where over 99\% of attacks are Brute Force Write operations.

\subsection{Feature Engineering}

We design 44 features categorized into three groups as shown in Table \ref{tab:features}. The feature extraction process follows a time-window approach with a duration of $T=15$ seconds.

\subsubsection{Key Feature: consecutive\_write\_max}
The feature \texttt{consecutive\_write\_max} ($cwm$) measures the maximum number of consecutive write operations within a time window. Formally, let $S = \{o_1, o_2, \dots, o_n\}$ be the sequence of operations in a window, where $o_i \in \{Read, Write\}$. The $cwm$ is defined as:
\begin{equation}
cwm = \max \{ k \mid \exists i: o_i, \dots, o_{i+k-1} = Write \}
\end{equation}
This feature captures the characteristic pattern of Brute Force Write attacks, where attackers rapidly send multiple write commands. Similar sequence-aware detection approaches have been explored in prior work \cite{ref_dpi_modbus}, demonstrating the importance of operation patterns in ICS anomaly detection \cite{ref_feature_eng}.

\subsubsection{Operational Pattern Features}
Beyond simple protocol fields, we derive semantic features such as \texttt{write\_without\_read\_ratio} ($wrr$), which represents the ratio of write operations that are not preceded by a corresponding read within the same window:
\begin{equation}
wrr = \frac{N_{write\_only}}{N_{total\_write}}
\end{equation}
where $N_{write\_only}$ is the count of write operations without prior reads. This captures deviations from typical "read-modify-write" industrial control loops.

\subsection{Dataset Characteristics Analysis}

A critical finding from our analysis is that the simple condition \texttt{consecutive\_write\_max > 0} can perfectly separate attack from normal traffic in both SCADA and IED scenarios:

\begin{itemize}
    \item SCADA: Attack samples with cwm>0 = 100\%, Normal samples with cwm>0 = 0\%
    \item IED: Attack samples with cwm>0 = 100\%, Normal samples with cwm>0 = 0\%
\end{itemize}

This reveals the relatively simple attack patterns in this dataset, which has important implications for method selection and result interpretation.

%==============================================================================
% IV. METHODOLOGY
%==============================================================================
\section{Methodology}

\subsection{Problem Formulation}
We formulate Modbus anomaly detection as a binary classification problem. Given a feature vector $\mathbf{x} \in \mathbb{R}^{44}$ extracted from a time window, the goal is to predict the label $y \in \{Normal, Attack\}$.

\subsection{Rule-based Baseline}
Based on our dataset analysis, we define a simple rule baseline:
\begin{equation}
\hat{y} = \begin{cases}
Attack & \text{if } cwm > 0 \\
Normal & \text{otherwise}
\end{cases}
\end{equation}
where $cwm$ denotes \texttt{consecutive\_write\_max}.

\subsection{Machine Learning Methods}
We evaluate five machine learning algorithms with specific hyperparameter configurations to ensure reproducibility:
\begin{itemize}
    \item \textbf{Random Forest (RF)}: An ensemble of 100 decision trees using Gini impurity and balanced class weights to mitigate the 21.7:1 imbalance.
    \item \textbf{XGBoost}: Gradient boosting with 100 estimators, a learning rate of 0.1, and a maximum depth of 6.
    \item \textbf{Decision Tree (DT)}: A single CART classifier with cost-complexity pruning.
    \item \textbf{Logistic Regression (LR)}: A linear model with L2 regularization and the liblinear solver.
    \item \textbf{MLP}: A multi-layer perceptron with two hidden layers (100, 50), ReLU activation, and the Adam optimizer.
\end{itemize}

All models are implemented using the Scikit-learn and XGBoost libraries. Training and inference are conducted on a workstation equipped with an Intel Core i7 processor and 16GB of RAM.

%==============================================================================
% V. EXPERIMENTS AND RESULTS
%==============================================================================
\section{Experiments and Results}

\subsection{Experimental Setup}
Data is split by PCAP files to prevent temporal leakage: 70\% for training, 15\% for validation, and 15\% for testing. We employ 5-fold cross-validation during the training phase to ensure model robustness. Performance is evaluated using Accuracy, Precision, Recall, F1-score, and Inference Time.

\subsection{Main Detection Results}

Table \ref{tab:main_results} presents the performance comparison of all methods.

\begin{table}[htbp]
\caption{Algorithm Performance Comparison}
\begin{center}
\begin{tabular}{lccccr}
\toprule
\textbf{Method} & \textbf{Acc.} & \textbf{Prec.} & \textbf{Rec.} & \textbf{F1} & \textbf{Time} \\
 & (\%) & (\%) & (\%) & (\%) & (ms) \\
\midrule
Simple Rule & 97.31 & 67.64 & \textbf{100.0} & 80.70 & 0.0001 \\
Random Forest & 99.21 & 94.84 & 90.83 & 92.79 & 0.0054 \\
XGBoost & \textbf{99.58} & \textbf{98.86} & 93.62 & \textbf{96.17} & 0.0010 \\
Decision Tree & 99.27 & 95.75 & 91.10 & 93.36 & 0.0002 \\
Logistic Reg. & 97.46 & 69.23 & 98.92 & 81.45 & 0.0002 \\
MLP & 98.18 & 97.96 & 69.06 & 81.01 & 0.0113 \\
\bottomrule
\end{tabular}
\label{tab:main_results}
\end{center}
\end{table}

\textbf{Key observations:}
\begin{itemize}
    \item XGBoost achieves the best F1-score (96.17\%) with highest precision (98.86\%)
    \item Simple rule achieves 100\% recall but only 67.64\% precision
    \item Tree-based models outperform the neural network (MLP)
\end{itemize}

\begin{figure}[htbp]
\centerline{\includegraphics[width=0.48\textwidth]{algorithm_performance_v2.png}}
\caption{Performance comparison of machine learning algorithms: (a) Detection performance metrics across different models, and (b) Inference time per sample in milliseconds.}
\label{fig:performance}
\end{figure}

\begin{figure}[htbp]
\centerline{\includegraphics[width=0.48\textwidth]{feature_importance_v2.png}}
\caption{Top 10 most important features identified by the Random Forest model for Modbus anomaly detection.}
\label{fig:importance}
\end{figure}

\begin{figure}[htbp]
\centerline{\includegraphics[width=0.48\textwidth]{two_layer_comparison_v2.png}}
\caption{Comparison of different detection architectures: Pure Machine Learning, Pure Rule-based, and the proposed Two-Layer hybrid architecture.}
\label{fig:two_layer}
\end{figure}

\begin{figure}[htbp]
\centerline{\includegraphics[width=0.48\textwidth]{ablation_comparison_v2.png}}
\caption{Ablation study results showing the impact of different feature categories (Protocol, Temporal, and Operational) on detection performance.}
\label{fig:ablation}
\end{figure}

\begin{figure}[htbp]
\centerline{\includegraphics[width=0.48\textwidth]{confusion_matrix_test.png}}
\caption{Confusion matrix of the XGBoost model on the test set, demonstrating high precision and recall.}
\label{fig:confusion}
\end{figure}

\subsection{ML vs Simple Rule Analysis}

Table \ref{tab:ml_gain} shows the performance improvement of ML methods over the simple rule baseline.

\begin{table}[htbp]
\caption{ML Improvement Over Simple Rule}
\begin{center}
\begin{tabular}{lccc}
\toprule
\textbf{Model} & \textbf{F1 Gain} & \textbf{Prec. Gain} & \textbf{Rec. Loss} \\
\midrule
Random Forest & +12.09\% & +27.20\% & -9.17\% \\
XGBoost & +15.47\% & +31.22\% & -6.39\% \\
\bottomrule
\end{tabular}
\label{tab:ml_gain}
\end{center}
\end{table}

ML methods trade a small recall decrease for substantial precision improvement, resulting in significant F1-score gains.

\subsection{Feature Importance Analysis}

Table \ref{tab:feature_importance} lists the top-10 important features. Notably, 6 of 10 features are related to write operations.

\begin{table}[htbp]
\caption{Top-10 Feature Importance}
\begin{center}
\begin{tabular}{clc}
\toprule
\textbf{Rank} & \textbf{Feature} & \textbf{Category} \\
\midrule
1 & fc\_read\_ratio & Protocol \\
2 & fc\_write\_ratio & Protocol \\
3 & consecutive\_write\_max & Operation \\
4 & consecutive\_write\_mean & Operation \\
5 & fc\_diversity & Protocol \\
6 & operation\_sequence\_entropy & Operation \\
7 & read\_write\_alternation & Operation \\
8 & packet\_size\_mean & Protocol \\
9 & packet\_size\_std & Protocol \\
10 & fc\_distribution\_entropy & Protocol \\
\bottomrule
\end{tabular}
\label{tab:feature_importance}
\end{center}
\end{table}

\subsection{Ablation Study: Feature Category Impact}

To understand the contribution of different feature categories, we conducted an ablation study. As shown in Table \ref{tab:ablation}, removing Operation Pattern features (DCS business logic) resulted in a slight improvement in F1-score (from 0.9279 to 0.9335 for RF).

\begin{table}[htbp]
\caption{Ablation Study Results (Random Forest)}
\begin{center}
\begin{tabular}{lccc}
\toprule
\textbf{Feature Set} & \textbf{Precision} & \textbf{Recall} & \textbf{F1-Score} \\
\midrule
All Features (Group 5) & 94.84\% & 90.83\% & 92.79\% \\
Protocol + Temporal (Group 4) & 95.31\% & 91.46\% & 93.35\% \\
Only Protocol (Group 1) & 95.21\% & 89.39\% & 92.21\% \\
Only Temporal (Group 2) & 93.28\% & 78.69\% & 85.37\% \\
Only Operation (Group 3) & 68.20\% & 98.38\% & 80.56\% \\
\bottomrule
\end{tabular}
\label{tab:ablation}
\end{center}
\end{table}

This unexpected result indicates that for the relatively straightforward attack patterns in the CIC Modbus 2023 dataset, complex operational features may introduce noise or lead to slight overfitting. Protocol-level features alone are highly effective for detection in these scenarios.

\subsection{Error Analysis}

Table \ref{tab:error_analysis} compares the error distributions.

\begin{table}[htbp]
\caption{Error Analysis}
\begin{center}
\begin{tabular}{lrrr}
\toprule
\textbf{Method} & \textbf{FN} & \textbf{FP} & \textbf{Total} \\
\midrule
Simple Rule & 0 & 532 & 532 \\
Random Forest & 21 & 185 & 206 \\
XGBoost & 5 & 108 & 113 \\
\bottomrule
\end{tabular}
\label{tab:error_analysis}
\end{center}
\end{table}

All 532 false positives from the simple rule originate from the benign scenario, caused by legitimate maintenance operations involving consecutive writes. Random Forest corrects 65.2\% (347/532) of these false positives, while XGBoost corrects 79.7\% (424/532).

\subsection{Scenario-wise Performance}

The simple rule achieves perfect detection (F1=1.0) in both SCADA and IED attack scenarios. False positives occur only in the benign scenario with an 8.96\% rate, corresponding to normal operational writes.

%==============================================================================
% VI. DISCUSSION
%==============================================================================
\section{Discussion}

\subsection{Key Findings}

\begin{enumerate}
    \item \textbf{Dataset Characteristics}: The CIC Modbus 2023 dataset exhibits relatively simple attack patterns, with \texttt{consecutive\_write\_max > 0} achieving perfect separation in attack scenarios. This finding has significant implications for understanding Modbus attack behaviors.
    
    \item \textbf{Feature Effectiveness}: Write operation-related features dominate the top importance rankings, indicating that unauthorized write operations are the primary attack vector in this dataset.
    
    \item \textbf{ML Value Proposition}: The primary value of ML methods lies in reducing false positives (65-80\% reduction) rather than improving detection rate. This is crucial for industrial operations where false alarms can disrupt production processes.
    
    \item \textbf{Algorithm Selection}: XGBoost provides the best balance between detection performance (F1=96.17\%) and computational efficiency (0.001ms inference), making it suitable for real-time industrial deployment.
\end{enumerate}

\subsection{Implications for Industrial Deployment}

Based on our findings and considerations for industrial system requirements \cite{ref_iiot_edge}, we provide deployment guidelines:

\begin{itemize}
    \item \textbf{Safety-Critical Systems}: For power grid substations and critical infrastructure where missed detections are unacceptable, deploy the simple rule baseline to ensure 100\% attack recall, accepting higher false positive rates that can be handled by human operators \cite{ref_lstm_scada}.
    
    \item \textbf{High-Availability Systems}: For manufacturing and process automation where both security and operational continuity are important, deploy XGBoost to balance detection accuracy with minimal false alarms \cite{ref_xgboost}.
    
    \item \textbf{Resource-Constrained Edge Devices}: For embedded industrial gateways and PLCs with limited computational resources, use the simple rule or Decision Tree for minimal overhead while maintaining effective detection \cite{ref_rf_ics}.
\end{itemize}

\subsection{Limitations and Future Work}

We acknowledge several limitations:
\begin{enumerate}
    \item \textbf{Attack Pattern Diversity}: Over 99\% of attacks in the dataset are Brute Force Write operations, limiting evaluation of detection capabilities for more sophisticated attack types such as stealthy data injection or protocol fuzzing.
    
    \item \textbf{Feature Separability}: The perfect separability achieved by a single feature suggests that the dataset may not fully represent the complexity of real-world industrial attacks.
    
    \item \textbf{Labeling Approach}: Our flow-based labeling strategy, while practical, may introduce minor label noise in edge cases.
\end{enumerate}

Future work should validate these methods on datasets with more diverse attack patterns and evaluate performance in operational industrial environments.

%==============================================================================
% VII. CONCLUSION
%==============================================================================
\section{Conclusion}

This paper presents a comprehensive feature-based analysis of the CIC Modbus Dataset 2023 and systematically evaluates anomaly detection methods for industrial control systems. Our key contributions include:

\begin{itemize}
    \item A systematic feature engineering approach with 44 features across protocol, temporal, and operation pattern categories, identifying \texttt{consecutive\_write\_max} as the critical discriminative feature for Modbus attack detection
    \item Quantitative evaluation demonstrating that XGBoost achieves 96.17\% F1-score with 0.001ms inference time, suitable for real-time industrial deployment
    \item Analysis showing that machine learning methods reduce false positives by 65-80\% compared to rule-based approaches, improving operational efficiency in industrial monitoring
    \item Practical deployment guidelines tailored to different industrial scenarios, balancing security requirements with operational constraints
\end{itemize}

Our findings provide valuable insights for enhancing the security and resilience of Modbus-based industrial systems. The methods and analysis presented in this paper can guide practitioners in deploying effective anomaly detection solutions and inform researchers using this dataset for future ICS security studies.

%==============================================================================
% REFERENCES
%==============================================================================
\begin{thebibliography}{00}

\bibitem{ref_iiot_security} A. Tange, M. De Donno, X. Fafoutis, and N. Dragoni, ``A systematic survey of industrial Internet of Things security: Requirements, attacks, AI-based solutions, and edge computing opportunities,'' \textit{Sensors}, vol. 23, no. 17, p. 7470, 2023.

\bibitem{ref_modbus_security} N. Goldenberg and A. Wool, ``Accurate modeling of Modbus/TCP for intrusion detection in SCADA systems,'' \textit{Int. J. Critical Infrastructure Protection}, vol. 6, no. 2, pp. 63--75, 2013.

\bibitem{ref_cic_dataset} Canadian Institute for Cybersecurity, ``CIC Modbus Dataset 2023,'' Univ. New Brunswick, 2023. [Online]. Available: https://www.unb.ca/cic/datasets/modbus-2023.html

\bibitem{ref_stuxnet} R. Langner, ``Stuxnet: Dissecting a cyberwarfare weapon,'' \textit{IEEE Security Privacy}, vol. 9, no. 3, pp. 49--51, May--Jun. 2011.

\bibitem{ref_blackenergy} R. M. Lee, M. J. Assante, and T. Conway, ``Analysis of the cyber attack on the Ukrainian power grid: Defense use case,'' Electricity Information Sharing and Analysis Center (E-ISAC), SANS ICS, Mar. 2016. [Online]. Available: https://ics.sans.org/media/E-ISAC\_SANS\_Ukraine\_DUC\_5.pdf

\bibitem{ref_wool} A. Kleinmann and A. Wool, ``Accurate modeling of the Siemens S7 SCADA protocol for intrusion detection and digital forensics,'' \textit{J. Digital Forensics, Security and Law}, vol. 9, no. 2, pp. 37--56, 2014.

\bibitem{ref_ml_ids} M. A. Upadhyay and K. P. Sharma, ``A review of research work on network-based SCADA intrusion detection systems,'' \textit{IEEE Access}, vol. 8, pp. 93561--93586, 2020.

\bibitem{ref_dl_ids} S. Potluri and C. Diedrich, ``Accelerated deep neural networks for enhanced intrusion detection system,'' in \textit{Proc. IEEE 21st Int. Conf. Emerging Technol. Factory Autom. (ETFA)}, Berlin, Germany, Sep. 2016, pp. 1--8.

\bibitem{ref_modbus_model} W. Gao and T. Morris, ``On cyber attacks and signature based intrusion detection for Modbus based industrial control systems,'' \textit{J. Digital Forensics, Security and Law}, vol. 9, no. 1, pp. 37--56, 2014.

\bibitem{ref_ics_ml} P. Garcia-Teodoro, J. Diaz-Verdejo, G. Maci{\'a}-Fern{\'a}ndez, and E. V{\'a}zquez, ``Anomaly-based network intrusion detection: Techniques, systems and challenges,'' \textit{Comput. Security}, vol. 28, no. 1--2, pp. 18--28, Feb.--Mar. 2009.

\bibitem{ref_rf_ics} Y. Zhu, J. Wang, and Z. Liang, ``Optimization of the Random Forest hyperparameters for power industrial control systems intrusion detection using an improved grid search algorithm,'' \textit{Appl. Sci.}, vol. 12, no. 20, p. 10456, Oct. 2022.

\bibitem{ref_xgboost} T. T. Le, H. Kim, H. Kang, and H. Kim, ``XGBoost for imbalanced multiclass classification-based industrial Internet of Things intrusion detection systems,'' \textit{Sustainability}, vol. 14, no. 14, p. 8707, Jul. 2022.

\bibitem{ref_lstm_scada} P. Radoglou-Grammatikis and P. Sarigiannidis, ``Securing the smart grid: A comprehensive compilation of intrusion detection and prevention systems,'' \textit{IEEE Access}, vol. 7, pp. 46595--46620, 2019.

\bibitem{ref_modbus_rules} J. Hong, C.-C. Liu, and M. Govindarasu, ``Integrated anomaly detection for cyber security of the substations,'' \textit{IEEE Trans. Smart Grid}, vol. 5, no. 4, pp. 1643--1653, Jul. 2014.

\bibitem{ref_dpi_modbus} M. Caselli, E. Zambon, and F. Kargl, ``Sequence-aware intrusion detection in industrial control systems,'' in \textit{Proc. 1st ACM Workshop Cyber-Physical System Security}, Singapore, Apr. 2015, pp. 13--24.

\bibitem{ref_iiot_edge} M. Aazam, S. Zeadally, and K. A. Harras, ``Deploying fog computing in industrial Internet of Things and Industry 4.0,'' \textit{IEEE Trans. Ind. Informat.}, vol. 14, no. 10, pp. 4674--4682, Oct. 2018.

\bibitem{ref_feature_eng} D. Hadžiosmanović, R. Sommer, E. Zambon, and P. H. Hartel, ``Through the eye of the PLC: Semantic security monitoring for industrial processes,'' in \textit{Proc. 30th Annu. Comput. Security Appl. Conf.}, New Orleans, LA, USA, Dec. 2014, pp. 126--135.

\end{thebibliography}

\end{document}
